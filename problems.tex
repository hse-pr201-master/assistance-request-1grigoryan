\documentclass[a4paper,12pt]{article}

\usepackage{cmap}
\usepackage{cool}
\usepackage{hyperref}
\usepackage{graphicx} 
\usepackage{subcaption}
\usepackage{wrapfig}
\usepackage{lastpage}
\usepackage{tikz}
\usepackage{pgfplots}
\usepackage{braket}
\usepackage{verbatim}
\usetikzlibrary{arrows}
\usetikzlibrary{calc,positioning,fit,backgrounds}
\usepackage{amsmath} 
\usepackage{amsfonts}
\usepackage[T2A]{fontenc}
\usepackage[utf8]{inputenc}	
\usepackage[english,russian]{babel}
\usepackage{geometry} 
\geometry{top=15mm}
\geometry{bottom=15mm}
\geometry{left=15mm}
\geometry{right=15mm}	
\usepackage{amssymb}
\usepackage{icomma} 
\usepackage{mathtext} 
\usepackage{mathrsfs}
\usepackage{mathtools}
\usepackage{physics}
\usepackage{fancyhdr}
\pagestyle{fancy}
\fancyhf{}
\renewcommand{\headrulewidth}{0,04mm}
\usepackage{hyperref}
\usepackage{mathtext} 
\setlength\parindent{0pt}
\usepackage{multicol}
\lhead{Теория вероятностей и математическая статистика}
\cfoot{\thepage} 
\makeatletter % сделать "@" "буквой", а не "спецсимволом" - можно использовать "служебные" команды, содержащие @ в названии
\renewcommand{\headrulewidth}{0,6mm} 

\renewcommand{\maketitle}{\begin{center}
		\noindent{\bfseries\scshape\LARGE\@title}\par
		\noindent {\large\itshape\@author}
		\vskip 2ex\end{center}}
\makeatother
	
\title{Задачки}

\begin{document}
	\maketitle
	
\section*{Задачка №1}

Я вот в 4 утра узнал, что есть фильм «Ковбои против пришельцев» (2015), так что задача будет по его мотивам (после экзамена по математической статистике посмотрю его даже).

Вообще, задача навеяна олимпиадой МЦНМО по теории вероятности для школьников, сам в ней участвовал, даже выиграл пару задачников.

\subsection*{Условие задачи}

Ковбой Буффало Билл и пришелец Слартибартфаст решили бросить кубик, чтобы узнать, кто из них круче. У кубика $\Factorial{n}$ граней ($n > 2$). 

Правила таковы: кубик им достался нечестный, они не знают истинных вероятностей выпадения граней, но предпологают, что вероятности эквивалентны номеру грани (то есть вероятность выпадения грани 6 в 6 раз больше, чем вероятность выпадения грани 1). В любом случае, они хотят сделать так, чтобы с помощью этого кубика можно было получить с равной вероятностью любое количество очков от 1 до $\Factorial{n}$. Как это можно сделать?

\subsection*{Одно из решений}

Будем подбрасывать кубик $n$ раз до тех пор, пока в комбинации не окажутся все различные числа. Теперь переведем вектор в такой вектор индексов, что он отсортирует изначальный вектор от меньшего к большему, например, для обычного игрального кубика $(4, 2, 6) \rightarrow (1, 0, 2)$. Данные комбинации должны появляться равновероятно (на примере обычного игрального кубика: так как вероятности выпадения комбинаций $(a, b, c), (a, c, b), (b, a, c), (b, c, a), (c, a, b), (c, b, a)$ для любой тройки очков равны, то и для всех троек данные вероятности будут равны). Соотнесем каждой комбинации свое количество очков, и получим равновероятные исходы.

\subsection*{Животные}

Ну, чем-то хорошим нужно было закончить эту страничку, так пусть это будут песели и котики. Да, вышло неровно, но так даже милее.

\begin{figure}[h!]
  \centering
  \begin{subfigure}[b]{0.3\linewidth}
    \includegraphics[width=\linewidth]{dog.jpg}
    \caption{Песель пытается создать 10000 симуляций, чтобы он мог оценить параметр $\theta$}
  \end{subfigure}
  \begin{subfigure}[b]{0.4\linewidth}
    \includegraphics[width=\linewidth]{cat.jpg}
    \caption{Котик наконец-то вывел нормальное распределение из двух словесных предпосылок и устало уснуг}
  \end{subfigure}
  \begin{subfigure}[b]{0.25\linewidth}
    \includegraphics[width=\linewidth]{dogs.jpg}
    \caption{Песики радостно едут сдавать экзамены офлайн на Покровку}
  \end{subfigure}
\end{figure}

\newpage

\section*{Задачка №2}

Эта задачка связана с новым сезоном Смешариков.

\subsection*{Условие задачи}

Ёжик захотел загадать три независимые и одинаково распределенные случайные величины: $X_i \sim Pois(\lambda) \hphantom {1} \forall i \in \{1, 2, 3\}$. У Ёжика специфичные представления о счастье: он счастлив, если будет верно следующее: $X_1^3 + X_2^3 = X_3^3$ при условии $ \abs{X_1 - X_2} = 1$.

Крош очень хочет, чтобы вероятность того, что Ёжик счастлив, была максимальна, ведь он его лучший друг. Крош может подобрать параметр $\lambda$ таким образом, чтобы вероятность счастья Ёжика была максимальна. Только вот Крош без понятия, как это сделать, поэтому он обратился к Лосяшу, который разбирается в науках математических.

Какое значение параметра $\lambda$ максимизирует вероятность счастья Ёжика?

\subsection*{Решение задачи}

Пусть без ограничения общности $X_2 = X_1 + 1$. Тогда будет верно следующее: $X_1^3 + (X_1 + 1)^3 = X_3^3$ (аналогично рассматривается случай $X_1 = X_2 + 1$, так как он симметричен). При $X_1 = 0$ мы получаем $X_3 = 1$. Пусть теперь $X_1 > 0$. Тогда мы получаем частный случай Великой теоремы Ферма: $a^3 + (a + 1)^3 = b^3$, и было доказано, что натуральных решений у данного уравнения не существует (честно хотел как-то доказать сам, но доказательство долгое). Отсюда нам подходят лишь две тройки: $(X_1, X_2, X_3) = (0, 1, 1)$ и  $(X_1, X_2, X_3) = (1, 0,  1)$. Пусть событие A - Ёжик счастлив. Вероятность выпадения этих троек равна $P(A) = 2 \cdot e^{-3\lambda} \cdot \lambda^2$. Запишем следующую задачу:

\begin{equation*}
\begin{cases}
p(\lambda) = 2 \cdot e^{-3\lambda} \cdot \lambda^2  \rightarrow \max{} \\
\lambda > 0
\end{cases}
\end{equation*}

Можем записать теперь FOC и SOC:
\begin{equation*}
\begin{cases}
\cfrac{\partial p}{\partial \lambda} = -6 \cdot e^{-3\lambda} \cdot \lambda^2 + 4 \cdot e^{-3\lambda} \cdot \lambda = e^{-3\lambda} \cdot \lambda \cdot (4 - 6 \cdot \lambda) = 0  \\
\cfrac{\partial^2 p}{\partial \lambda^2} = e^{-3\lambda} \cdot (18\cdot \lambda^2 - 24 \cdot \lambda + 4) < 0 \\
\lambda > 0
\end{cases}
\end{equation*}

Отсюда $\lambda^* = \cfrac{2}{3}$. Вероятность того, что Ёжик будет счастлив, равна $\cfrac{8}{9\cdot e^2} \approx 0.12$.

\end{document}